\section*{Conclusions}
The first combined PIV-PLIF measurements in a dual-mode scramjet combustor are presented, giving new insight into an unsteady flame that is stabilized on a cavity in a high speed, premixed flow, and contributing to experimental databases suitable for validation of computational models for such flows.  The flow geometry is explicitly sized to be suitable for DNS modeling. The cavity flow is characterized by an unsteady shear layer and a recirculation zone that exhibits a long period cycle, with scales much larger than the eddy sizes associated with the shear layer above the cavity. This cycle was identified by categorizing PLIF images by common flamefront traits, then sorting them based on the companion velocimetry and physical plausibility. This proposed cycle starts with a trapped vortex that accumulates mass and heat in the cavity. The accumulation is unsustainable and the trapped vortex splits, triggering eventual detachment of the shear layer from the cavity ramp and the ejection of combustion products. As these ejected combustion products move out of the cavity, the flame is broken and products are shed into the duct flow. The shear layer then reattaches and resets the cavity to the trapped vortex state. These observations are consistent with previous work on similar cavity-stabilized flames using non-simultaneous PIV and PLIF. It is speculated that this cavity cycle is phase locked to thermoacoustic oscillations arising from acoustic interaction between the shock train in the inlet isolator and the thermal throat downstream of the cavity.  More experiments with high speed diagnostics are needed to better understand the coupling of the cavity-stabilized combustion dynamics with global instabilities of the engine. Large eddy simulations will also deliver valuable insights into the oscillations across wide temporal and spatial scales. Improving understanding of dual-mode scramjet combustion dynamics in this manner may lead to innovative flameholder geometries and active or passive dampeners for operational scramjet propulsion systems.
