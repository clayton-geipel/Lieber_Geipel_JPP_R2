\begin{abstract}
Combined particle image velocimetry and hydroxyl planar laser induced fluorescence measurements are conducted in a dual-mode combustor operated in ramjet mode, with premixed ethylene-air combustion stabilized by a cavity. The cavity is scaled to be suitable for direct numerical simulations. Simultaneous measurements of instantaneous flow velocities and locations of flame products enable deeper insights into turbulence-flame interactions. The shear layer originating at the cavity leading edge generates periodic vortical structures and a wrinkled flame. Under the shear layer, an unsteady recirculation zone promotes mixing of cavity-generated combustion products with the free stream. A description of periodic, unsteady combustion in the cavity is proposed. The cycle starts with a trapped recirculation vortex in the cavity near the leading edge which grows as it accumulates mass and heat. This unsustainable state triggers splitting of the vortex and detachment of the shear layer from the cavity ramp. Combustion products and the downstream vortex are ejected downstream from the cavity and the flame is broken. The cycle is completed when the shear layer reattaches and again forms a trapped recirculation vortex. This unsteady combustion cycle may be phase-coupled with a global flame instability such as thermoacoustic oscillations.
\end{abstract}