\section*{Introduction}
\lettrine{U}{nderstanding} the fundamental physics governing cavity flameholders is critical to the development of high-speed air-breathing propulsion systems such as hydrocarbon-fueled dual-mode scramjets. Hydrocarbon fuels are easier to store and handle than hydrogen and therefore considered more practical. In the absence of a flameholder in the engine, the longer reaction times of these fuels inhibit the complete release of their chemical energy before exiting the scramjet. Cavities solve this problem by locally increasing the residence time, creating a shear layer and flow recirculation \citep{Ben-YakarHanson2001}. Combustion radicals are produced in the cavity before being diffused through turbulent fluctuations across the shear layer and intermittently ejected into the main flow \citep{Kirik2017}. In addition to applications in scramjet combustors, cavity flameholders enable fundamental research on high-speed combustion by establishing in a premixed free stream a flame with a relatively large normal velocity. The intricate interactions between the flow unsteadiness and chemical reactions are however not fully understood and warrant further investigation. In particular, combustion instabilities in premixed dual-mode scramjet flowpaths with a cavity flameholder are an area of continuing research \citep{WangWangSun2014}. The problem is critical to enabling operational engines and reliable computational fluid dynamics (CFD) simulations.
 
Scramjet combustors with flush or ramp injectors have been extensively studied \citep{RockwellGoyneRiceEtAl2014, LaurenceLieberSchrammEtAl2015, LePichonLaverdant2016, FangHong2018}. High-speed cavity flameholders became a topic of interest more recently. Tuttle et al.
\cite{TuttleCarterHsu2014} published the first study of these flameholders operating in scram-mode by implementing particle imaging velocimetry (PIV) and comparing results to a separate planar laser-induced fluorescence (PLIF) study in order to investigate the dependence of velocity behavior on the heat release in the cavity and shear layer. These authors found a strong coupling between various velocimetry statistics and heat release rates, noting the value simultaneous PIV and PLIF would hold in order to clarify coupling between heat release and velocity.
Research on premixed and dual-mode scramjet combustors with a cavity flameholder is particularly recent; the only study known to the authors \cite{KirikGoyneMcDanielEtAl2017} was conducted on the same facility as the present work but with a larger cavity and flowpath. The investigators characterized premixed cavity aerodynamics and its relation to both the shock train upstream location and the fuel equivalence ratio by conducting PIV and PLIF of the hydroxyl (OH) radical as separate experiments. In particular, their comparison of the peak fluctuations in velocities and relative OH levels suggest an unsteady combustion process whereby pockets of combustion products produced near the ramp are intermittently ejected into the main duct flow.


Chemiluminescence imaging of the CH* radical has been performed \cite{AllisonFredericksonKirikEtAl2017}  on a similar cavity flameholder flowpath in the University of Virginia Supersonic Combustion Facility (UVaSCF). That study suggested a characteristic frequency of 340 Hz in the CH* signal and flame brush, which is likely due to acoustic waves between the shock train and the thermal throat. Phase-coupled addition of energy to these waves occurs due to heat addition by combustion in the flow between the cavity and the thermal throat. 
Stable combustion is desired for operational application on aircraft, hence the mechanisms driving combustion oscillations warrant further investigation.
\citet{WangWangSun2014} reviews recent progress on combustion oscillations in dual-mode scramjets and highlights their variety and complexity. Thermoacoustic oscillations are on the order of hundreds of Hz while the cavity shear layer oscillates at higher frequencies, particularly for low heat release rates.  
\citet{LinJacksonBehdadniaEtAl2010} used a high-frequency pressure sensor and theoretical/numerical analysis to study oscillations in a dual-mode, non-premixed scramjet flowpath. For the conditions closest to the present work with a single-sided injection from the cavity wall, a dominant frequency of 368 Hz is measured. The oscillation is attributed to acoustic-convective wave interactions between the terminal shock and the thermal throat and between the fuel injectors and the thermal throat. In this case, the oscillation in local heat release rate sends an acoustic wave upstream, which modifies fuel penetration and mixing. 
Hybrid large eddy simulation (LES)/Reynolds-averaged Navier-Stokes simulations by \citet{ramesh2015large} of a dual-mode premixed flowpath with a cavity flameholder observed a 357 Hz cyclic combustion pattern. While the non-satisfaction of the Rayleigh criterion and the large amplitude of the oscillations hint at a computational aberration, their work nonetheless gives valuable insights into interactions between the shock train and the cavity-stabilized flame. These authors report an oscillation in heat release. Mass accumulates periodically in the cavity, reducing heat release, increasing cavity pressure, and causing a normal shock wave to propagate upstream. Increasing heat release is associated with a positive mass flow rate due to ejection events and higher duct flow velocities due to thermoacoustic fluctuations.
Non-simultaneous PIV and PLIF measurements on single- and dual-mode scramjet combustors also observed hints at an oscillatory combustion pattern. 
Tuttle et al. \cite{TuttleCarterHsu2014} suggest an intermittently-stable shear layer flame driving cavity combustion dynamics in a supersonic flow, whereby heat release influences the shear layer impingement location and the formation of the recirculation-zone eddy structures.
Kirik \cite{Kirik2017} observed that locations with the strongest turbulent fluctuations are adjacent to locations with high OH concentration fluctuations, hinting at the intermittent release of high-temperature products and radical species from the cavity.  
Tuttle et al. \cite{TuttleCarterHsu2014} and Kirik \cite{Kirik2017} noted that simultaneous PIV and PLIF would help understanding of the instantaneous mechanism behind the unsteady combustion process. 


%Cavities without combustion have been extensively studied and provide additional material to interpret the mechanisms behind reacting cavity instabilities. Gharib and Roshko \cite{GharibRoshko1987} report on a locked vortex configuration within the cavity corresponding to stable flow. This configuration is however established at the cost of low mass exchange between the cavity and the duct flow. 
%Ben-Yakar and Hanson \citep{Ben-YakarHanson2001}, in a comprehensive review about cavity flows, described the phenomenon of longitudinal oscillations in non-combusting rectangular cavity flow. These oscillations are driven by shear layer unsteadiness adding and removing mass at the cavity closeout. When the freestream flow enters the cavity at the impingement point, a compression wave propagates upstream, impacts the front wall, and generates small vortices at the leading edge. These eddies are in turn convected downstream, being amplified along the way. The cycle ends with vortex shedding, ejecting the mass initially added. Suppression of these cavity oscillations may be achieved with an oblique closeout wall to prevent the reflection of acoustic waves, as implemented in the present flowpath geometry.
%Low speed reacting flows in closed cavities were studied by \cite{XavierVandelGodardEtAl2016}, who resolved a full instability cycle with combined kHz PIV-PLIF. These authors highlighted the complex coupling between shear layer instabilities and the observed pulsed combustion regime. A stable combustion mode was reached at an increased free stream momentum relative to cavity injection momentum, which contains the cavity flow and constrains combustion within the shear layer. They suggested that large velocity fluctuations between the cavity and the main flow cause full shear layer modifications.

While separate PIV and PLIF measurements can give important insights by relating key turbulence characteristics with combustion-related statistics, they are limited by their temporally-uncorrelated nature. As a result, the analysis is constrained to relating independent statistical quantities. In addition, small variations within or between runs in total pressure, total temperature, or equivalence ratio further add to the limitations of this approach. In contrast, PIV-PLIF enables measurements that are correlated in time and space, providing deeper insight into the coupling between combustion physics and fluid mechanics \citep{TanahashiMurakamiChoiEtAl2005, GambaClemensEzekoye2013, XavierVandelGodardEtAl2016, FangHong2018}. 
The fundamental added value of PIV-PLIF is the knowledge of the distribution of products and reactants relative to the velocity fields for each instantaneous measurement, which is exploited in several ways. First, the velocity field can be divided into products and reactants domains --- i.e. conditional velocities --- from which stem conditional statistics, e.g. conditional means. 
Second, the instantaneous OH region boundaries from the PLIF signal approximate the local flame curvature and angle, which can then be related to the flow swirl, turbulence intensity, and direction. 
Third, knowledge of both the instantaneous flow dynamics and flame location provides complementary knowledge to characterize the flame anchoring process in terms of temporal dynamics. 
By leveraging the unique capabilities of simultaneous PIV-PLIF listed above, the present work aims to quantify the key characteristics of the chemically-reacting flow field of dual-mode cavity flameholders.

This work provides the first dataset of correlated velocimetry and OH distributions including conditional velocimetry statistics for numerical studies, to better understand flameholding mechanisms at play in a cavity-stabilized flame in a dual-mode scramjet operating in ramjet mode. In this mode, a shock train upstream of the combustor terminates in a normal shock, resulting in subsonic flow in the combustor. In particular, the mass transfer across the cavity boundary, the conditional velocimetry are investigated through mean and RMS velocity fields, the turbulent kinetic energy, and the turbulence intensity for both the full velocity field and conditional velocities; as well as the characterization of the cavity in terms of the recirculation regions,
shear layer impingement location, flame anchoring location, and temporal dynamics. In addition, the mass transfer of products and reactants at the cavity interface will be estimated based on the average transverse conditional velocities. 

The present work is part of a larger collaborative effort to deepen the fundamental understanding of cavity-stabilized combustion including its relation with flow compressibility and heat release rates. To that end, simultaneous PIV and OH PLIF, hereafter PIV-PLIF, are conducted on a high-speed reacting cavity flow. PIV-PLIF measurement has the advantages over separate measurement that flow-chemistry coupling is investigated with instantaneous measurements and complementary validation metrics are provided to companion CFD work. In particular, direct numerical simulations (DNS, discussed in \cite{rauch2018dns}) are being conducted parallel to the present experimental work to gain insights at the finest spatio-temporal scales. 

Because the cavities and flowpaths investigated previously were too large for feasible DNS computational times, the UVaSCF \cite{KraussJ.McDaniel1992, McDanielGoyneBrynerEtAl2005} has been modified to host a small-scale cavity flameholder as detailed in \cite{LieberGoyneRockwellEtAl2018a}. This cavity provides a smaller computational domain for CFD work, such that DNS is feasible with currently-available computing resources \cite{rauch2018dns}. The results presented below therefore constitute the first published reference dataset on a high-speed reacting cavity flow that may be computed using DNS. The inflow to the cavity has been characterized with high-resolution PIV in a separate study \cite{LieberThesis}.

The paper is structured as follows. First, the combined setup and post-processing are summarized. Second, the DNS-friendly, scaled-down, and immersed cavity flow is characterized with statistics and key values provided as references for CFD work. Finally, combustion dynamics are investigated with instantaneous measurements and a hypothetical combustion cycle is suggested for dual-mode scramjet cavity flameholder instability.